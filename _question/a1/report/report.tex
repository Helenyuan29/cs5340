%%%%%%%%%%%%%%%%%%%%%%%%%%%%%%%%%%%%%%%%%
% University/School Laboratory Report
% LaTeX Template
% Version 3.1 (25/3/14)
%
% This template has been downloaded from:
% http://www.LaTeXTemplates.com
%
% Original author:
% Linux and Unix Users Group at Virginia Tech Wiki 
% (https://vtluug.org/wiki/Example_LaTeX_chem_lab_report)
%
% License:
% CC BY-NC-SA 3.0 (http://creativecommons.org/licenses/by-nc-sa/3.0/)
%
%%%%%%%%%%%%%%%%%%%%%%%%%%%%%%%%%%%%%%%%%

%----------------------------------------------------------------------------------------
%	PACKAGES AND DOCUMENT CONFIGURATIONS
%----------------------------------------------------------------------------------------

\documentclass[paper=a4, fontsize=11pt]{scrartcl} % A4 paper and 11pt font size

\usepackage[version=3]{mhchem} % Package for chemical equation typesetting
\usepackage{siunitx} % Provides the \SI{}{} and \si{} command for typesetting SI units
\usepackage{graphicx} % Required for the inclusion of images
%\usepackage{natbib} % Required to change bibliography style to APA
\usepackage{amsmath} % Required for some math elements 
\usepackage{wrapfig}
\usepackage{url}
\usepackage{hyperref}


\setlength\parindent{0pt} % Removes all indentation from paragraphs

\renewcommand{\labelenumi}{\alph{enumi}.} % Make numbering in the enumerate environment by letter rather than number (e.g. section 6)

%\usepackage{times} % Uncomment to use the Times New Roman font

%----------------------------------------------------------------------------------------
%	DOCUMENT INFORMATION
%----------------------------------------------------------------------------------------

\title{CS5242 Assignment Report} % Title

%\author{Group 16} % Author name

\date{\today} % Date for the report

\begin{document}

\maketitle % Insert the title, author and date

\begin{center}
\begin{tabular}{l r}
%Date Performed: & January 1, 2012 \\ % Date the experiment was performed
Partners: & Kyaw Zaw Lin(E0218306) \\ % Partner names
& Rahul Soni(E???????) \\

\end{tabular}
\end{center}

% If you wish to include an abstract, uncomment the lines below
% \begin{abstract}
% Abstract text
% \end{abstract}

%----------------------------------------------------------------------------------------
%	SECTION 1
%----------------------------------------------------------------------------------------

\section{Assignment 1}

Gibbs sampling is one of the MCMC algorithms. Given a set of random variables, $\{z_i:i=1,...,M\}$, and  time steps $\tau=1,...T$,
\begin{itemize}
	\item $z_1^{\{\tau+1\}} \sim  p(z_1|z_2^{\{\tau\}},...,z_M^{\{\tau\}})$
	\item $z_2^{\{\tau+1\}} \sim  p(z_2|z_2^{\{\tau+1\}},...,z_M^{\{\tau\}})$
	\item $...$
	\item $z_M^{\{\tau+1\}} \sim  p(z_M|z_2^{\{\tau+1\}},...,z_{M-1}^{\{\tau+1\}})$
\end{itemize}

In other words, at each time step, we sample from the full condtional of one variable, also written $p(z_i|z_{-i})$, which is the probability of $z_i$ given all others. Then we update the state of variable and proceed until time step $T$. The initial samples are discarded until the Markov chain has burned in, or entered its stationary distribution. In our implementation, we discard initial 25\% of the samples. 

For image denoising, we apply gibbs sampling on the pairwise Ising CRF model where pairwise potentials $\psi_{st}(x_s,x_t)$ are given by $exp(Jx_sx_t)$ where $ x_s,x_t \in \{-1,+1\}$.


\begin{align} 
\begin{split}
p(x_t|x_{-t},\theta) & \propto \prod_{s\in nbr(t)} \psi_{st}(x_s,x_t)\\
%					 & \propto \prod_{s\in nbr(t)} exp(Jx_sx_t)\\
\end{split}					
\end{align} 

The model can be further extended by incorporting local evidence $\psi_t(x_t)$. 

\begin{align} 
\begin{split}
	p(x_t|x_{-t},y,\theta) & \propto \psi_t(x_t) \prod_{s\in nbr(t)} \psi_{st}(x_s,x_t)\\
%	& \propto \prod_{s\in nbr(t)} exp(Jx_sx_t)\\
\end{split}						
\end{align} 

By using gaussian observation model where $\psi_t(x_t) = \mathcal{N}(y_t|x_t,\sigma^2)$, and normalizing we obtain the prbability of a particular pixel being noise:


\begin{align} 
\begin{split}
p(x_t=+1|x_{-t},y,\theta) & = \dfrac{\psi_t(+1) \prod_{s\in nbr(t)} \psi_{st}(x_t=+1,x_s)}{\psi_t(+1) \prod_{s\in nbr(t)} \psi_{st}(x_t=+1,x_s) + \psi_t(-1)\prod_{s\in nbr(t)} \psi_{st}(x_t=-1,x_s)}\\
& = \dfrac{\psi_t(+1) exp[J\sum_{s\in nbr(t)}{x_s}]}{\psi_t(+1) exp[J\sum_{s\in nbr(t)}{x_s}]+\psi_t(-1) exp[-J\sum_{s\in nbr(t)}{x_s}]}\\
\end{split}						
\end{align} 





\bibliographystyle{ieeetr}

\bibliography{report} 

%----------------------------------------------------------------------------------------


\end{document}